\section{Einleitung}\label{einleitung}
Der fortwährende Wandel innerhalb der IT hat dazu beigetragen, dass auch SAP sich mit neuen UI Technologien beschäftigt. Im Zuge der neuen UI Strategie wurde von der SAP AG ein Bündel an JavaScript (JS) Bibliotheken geschnürt, mit welchem Entwickler in die Lage versetzt werden sollen schnell und unkompliziert Applikationen entwickeln zu können. Im Hintergrund kann die Anwendung von einem beliebigem SAP-System mit Daten bedient werden. Die SAP AG hat bei der Entwicklung des Frameworks auf aktuelle Open-Source Technologien gesetzt. Entstanden ist diese Arbeit in Zusammenarbeit mit der abat AG, einem SAP Beratungsunternehmen mit Spezialisierung in der Automotive und Logistik Branche.\par Für viele Kunden werden Eigenentwicklungen im SAP Umfeld benötigt. Aus diesem Umstand ist das Thema zur prototypischen Implementierung einer SAP UI5-Applikation im SAP Umfeld entstanden. Das überwiegend eingesetzte SAP GUI ist optisch nicht sehr ansprechend. Eigenentwicklungen müssen sich an die Design Richtlinien der SAP halten und umständlich an die Funktionen und Möglichkeiten des SAP GUI angepasst werden. Soll beispielsweise die Unternehmenssoftware an das Corporate Design angepasst werden, aus Gründen der Produktivitätssteigerung oder dem Streben nach einer höheren Nutzerzufriedenheit, ist dies mit dem Standard SAP GUI schwer bis gar nicht umsetzbar.\par Ziel des Autors war es, eine prototypische Applikation im Master/Detail Layout mit Hilfe der SAP UI5-Bibliotheken innerhalb eines SAP Umfeldes zu implementieren. Außerdem sollte eine Analyse der eingesetzten UI-Objekte durchgeführt werden. Dazu war eine technische Auseinandersetzung mit den Technologien, die innerhalb des SAP UI5-Frameworks verwendet werden, nötig.\par Die Arbeit ist nach folgender Struktur aufgebaut. In Kapitel 2 werden die, in der Implementation verwendeten, Technologien erläutert. Es wird sehr detailliert auf die Regeln und Besonderheiten der einzelnen Sprachen eingegangen. Dies ist nötig, um die in Kapitel 3 beschriebene Implementierung des Prototypen flüssig gestalten zu können. Des Weiteren werden die benutzten Tools erläutert. Am Ende des Kapitels wurden alle Schritte durchgeführt, die nötig waren, um den Prototyp auf einem SAP-System bereitzustellen. In Kapitel 4, der Analyse, werden der Prototyp und die SAP UI5-Bibliotheken betrachtet. Die Zusammenhänge und Besonderheiten der Bibliotheken werden aus Entwicklungssicht dargestellt.