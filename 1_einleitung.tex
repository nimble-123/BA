\section{Einleitung}\label{einleitung}
Der fortwährende Wandel innerhalb der IT hat dazu beigetragen das auch die SAP sich mit neuen UI Technologien beschäftigt. Im Zuge der neuen UI Strategie wurde von der SAP AG ein Bündel an JavaScript Bibliotheken geschnürt mit welchem Entwickler in die Lage versetzt werden sollen schnell und unkompliziert Applikationen entwickeln zu können die im Hintergrund von einem beliebigem SAP System mit Daten bedient werden können. Die SAP hat nicht versucht das Rad neu zu erfinden. Sie hat bei der Entwicklung des Bibliotheken Bündels auf aktuelle Open-Source Technologien gesetzt. Entstanden ist diese Arbeit in Zusammenarbeit mit der abat AG. Einem SAP Beratungsunternehmen mit Spezialisierung auf den Automotive Bereich. Für viele Kunden werden Eigenentwicklungen im SAP Umfeld benötigt, aus diesem Umstand ist das Thema zur prototypischen Implementierung einer SAP UI5 Applikation im SAP Umfeld entstanden. Das überwiegend eingesetzte SAP GUI ist optisch nicht sehr ansprechend. Eigenentwicklungen müssen sich an die Design Richtlinien der SAP halten und umständlich an die Funktionen und Möglichkeiten des SAP GUI angepasst werden. Soll beispielsweise die Unternehmenssoftware an das Corporate Design angepasst werden, aus Gründen der Produktivitätssteigerung oder dem Streben nach einer höheren Nutzerzufriedenheit, ist dies mit dem Standard SAP GUI schwer bis gar nicht umsetzbar.\\
Ziel des Autors war es eine prototypische Applikation im Master/Detail Layout mit Hilfe der SAP UI5 Bibliotheken innerhalb eines SAP Umfeldes zu implementieren. Außerdem sollte eine Analyse der eingesetzten UI Objekte durch geführt werden. Dazu war auch eine technische Auseinandersetzung mit den Technologien die innerhalb des SAP UI5 Frameworks verwendet werden nötig.\\
Nach diesem Kapitel werden in Kapitel 2 die, in der Implementation verwendeten, Technologien erläutert. Es wird sehr detailliert auf die Regeln und Besonderheiten der einzelnen Sprachen eingegangen. Dies ist nötig um die in Kapitel 3 beschriebene Implementierung des Prototypen flüssig und nicht zu aufgebläht mit Rand Erklärungen gestalten zu können.  Des Weiteren werden die benutzten Tools kurz näher gebracht. Am Ende des Kapitels sind alle Schritte durchgeführt worden, die nötig waren um den Prototyp auf einem SAP System bereitzustellen. In Kapitel 4, der Analyse, werden der Prototyp und die SAP UI5 Bibliotheken näher betrachtet. Ihre Zusammenhänge und Besonderheiten aus Entwicklungssicht werden dargestellt.