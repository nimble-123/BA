\documentclass[12pt,a4paper,bibliography=totocnumbered,listof=totocnumbered]{scrartcl}
\usepackage[ngerman]{babel}
\usepackage[utf8]{inputenc}
\usepackage{amsmath}
\usepackage{amsfonts}
\usepackage{amssymb}
\usepackage{graphicx}
\usepackage{fancyhdr}
\usepackage{tabularx}
\usepackage{geometry}
\usepackage{setspace}
\usepackage[right]{eurosym}
\usepackage[printonlyused]{acronym}
\usepackage{subfig}
\usepackage{floatflt}
\usepackage[usenames,dvipsnames]{color}
\usepackage{colortbl}
\usepackage{paralist}
\usepackage{array}
\usepackage{titlesec}
\usepackage{parskip}
\usepackage[right]{eurosym}
\usepackage{picins}
\usepackage[subfigure,titles]{tocloft}
\usepackage[pdfpagelabels=true]{hyperref}

\usepackage{listings}
\lstset{basicstyle=\footnotesize, captionpos=b, breaklines=true, showstringspaces=false, tabsize=2, frame=lines, numbers=left, numberstyle=\tiny, xleftmargin=2em, framexleftmargin=2em}
\makeatletter
\def\l@lstlisting#1#2{\@dottedtocline{1}{0em}{1em}{\hspace{1,5em} Lst. #1}{#2}}
\makeatother

\geometry{a4paper, top=27mm, left=30mm, right=20mm, bottom=35mm, headsep=10mm, footskip=12mm}

\hypersetup{unicode=false, pdftoolbar=true, pdfmenubar=true, pdffitwindow=false, pdfstartview={FitH},
	pdftitle={Bachelorarbeit},
	pdfauthor={Nils Lutz},
	pdfsubject={Bachelorarbeit},
	pdfcreator={\LaTeX\ with package \flqq hyperref\frqq},
	pdfproducer={pdfTeX \the\pdftexversion.\pdftexrevision},
	pdfkeywords={Bachelorarbeit},
	pdfnewwindow=true,
	colorlinks=true,linkcolor=black,citecolor=black,filecolor=magenta,urlcolor=black}
\pdfinfo{/CreationDate (D:20110620133321)}

\begin{document}

\titlespacing{\section}{0pt}{12pt plus 4pt minus 2pt}{-6pt plus 2pt minus 2pt}

% Kopf- und Fusszeile
\renewcommand{\sectionmark}[1]{\markright{#1}}
\renewcommand{\leftmark}{\rightmark}
\pagestyle{fancy}
\lhead{}
\chead{}
\rhead{\thesection\space\contentsname}
\lfoot{ARBEITSTITEL\newline ZWEITE REIHE}
\cfoot{}
\rfoot{\ \linebreak Seite \thepage}
\renewcommand{\headrulewidth}{0.4pt}
\renewcommand{\footrulewidth}{0.4pt}

% Vorspann
\renewcommand{\thesection}{\Roman{section}}
\renewcommand{\theHsection}{\Roman{section}}
\pagenumbering{Roman}

% ----------------------------------------------------------------------------------------------------------
% Titelseite
% ----------------------------------------------------------------------------------------------------------
\thispagestyle{empty}
\begin{center}
	\includegraphics[scale=0.7]{images/fh_whv_big.jpg}\\
	\vspace*{2cm}
	\Large
	\textbf{Jade Hochschule}\\
	\textbf{Management, Information \& Technologie}\\
	\textbf{Wirtschaftsinformatik}\\
	\vspace*{2cm}
	\Huge
	\textbf{Bachelorarbeit}\\
	\vspace*{0.5cm}
	\large
	über das Thema\\
	\vspace*{1cm}
	\textbf{ARBEITSTITEL}\\
	\vspace*{2cm}
	
	\vfill
	\normalsize
	\newcolumntype{x}[1]{>{\raggedleft\arraybackslash\hspace{0pt}}p{#1}}
	\begin{tabular}{x{6cm}p{7.5cm}}
		\rule{0mm}{5ex}\textbf{Autor:} & Nils Lutz\newline info@nilslutz.de \\ 
		\rule{0mm}{5ex}\textbf{Erstprüfer:} & Prof. Dr.-Ing. Hergen Pargmann \\ 
		\textbf{Zweitprüfer:} & Prof. Dr. Harald Schallner \\ 
		\rule{0mm}{5ex}\textbf{Abgabedatum:} & 25.01.2015  \\ 
	\end{tabular} 
\end{center}
\pagebreak

% ----------------------------------------------------------------------------------------------------------
% Abstract
% ----------------------------------------------------------------------------------------------------------
\setcounter{page}{1}
\onehalfspacing
\titlespacing{\section}{0pt}{12pt plus 4pt minus 2pt}{2pt plus 2pt minus 2pt}
\rhead{KURZFASSUNG}
\section{Kurzfassung}
Lorem ipsum dolor sit amet, consetetur sadipscing elitr, sed diam nonumy eirmod tempor invidunt ut labore et dolore magna aliquyam erat, sed diam voluptua. At vero eos et accusam et justo duo dolores et ea rebum. Stet clita kasd gubergren, no sea takimata sanctus est Lorem ipsum dolor sit amet. Lorem ipsum dolor sit amet, consetetur sadipscing elitr, sed diam nonumy eirmod tempor invidunt ut labore et dolore magna aliquyam erat, sed diam voluptua. At vero eos et accusam et justo duo dolores et ea rebum. Stet clita kasd gubergren, no sea takimata sanctus est Lorem ipsum dolor sit amet. 

\vspace{-1,2em}
\titlespacing{\section}{0pt}{12pt plus 4pt minus 2pt}{-6pt plus 2pt minus 2pt}
\section*{Abstract}
Das ganze auf Englisch.
\pagebreak

% ----------------------------------------------------------------------------------------------------------
% Verzeichnisse
% ----------------------------------------------------------------------------------------------------------
% TODO Typ vor Nummer
\renewcommand{\cfttabpresnum}{Tab. }
\renewcommand{\cftfigpresnum}{Abb. }
\settowidth{\cfttabnumwidth}{Abb. 10\quad}
\settowidth{\cftfignumwidth}{Abb. 10\quad}

\titlespacing{\section}{0pt}{12pt plus 4pt minus 2pt}{2pt plus 2pt minus 2pt}
\singlespacing
\rhead{INHALTSVERZEICHNIS}
\renewcommand{\contentsname}{II Inhaltsverzeichnis}
\phantomsection
\addcontentsline{toc}{section}{\texorpdfstring{II \hspace{0.35em}Inhaltsverzeichnis}{Inhaltsverzeichnis}}
\addtocounter{section}{1}
\tableofcontents
\pagebreak
\rhead{VERZEICHNISSE}
\listoffigures
\pagebreak
\listoftables
%\pagebreak
\renewcommand{\lstlistlistingname}{Listing-Verzeichnis}
{\labelsep2cm\lstlistoflistings}
\pagebreak

% ----------------------------------------------------------------------------------------------------------
% Abkürzungen
% ----------------------------------------------------------------------------------------------------------
\section{Abkürzungsverzeichnis}
\begin{acronym}[OSGi] % längste Abkürzung steht in eckigen Klammern
	\setlength{\itemsep}{-\parsep} % geringerer Zeilenabstand
	\acro{OSGi}{Open Service Gateway initiative}
	\acro{JSP}{Java Server Pages}
	\acro{BSP}{Business Server Pages}
	\acro{SPP}{Spare Parts Planning}
\end{acronym}
\newpage

% ----------------------------------------------------------------------------------------------------------
% Inhalt
% ----------------------------------------------------------------------------------------------------------
% Abstände Überschrift
\titlespacing{\section}{0pt}{12pt plus 4pt minus 2pt}{-6pt plus 2pt minus 2pt}
\titlespacing{\subsection}{0pt}{12pt plus 4pt minus 2pt}{-6pt plus 2pt minus 2pt}
\titlespacing{\subsubsection}{0pt}{12pt plus 4pt minus 2pt}{-6pt plus 2pt minus 2pt}

% Kopfzeile
\renewcommand{\sectionmark}[1]{\markright{#1}}
\renewcommand{\subsectionmark}[1]{}
\renewcommand{\subsubsectionmark}[1]{}
\lhead{Kapitel \thesection}
\rhead{\rightmark}

\onehalfspacing
\renewcommand{\thesection}{\arabic{section}}
\renewcommand{\theHsection}{\arabic{section}}
\setcounter{section}{0}
\pagenumbering{arabic}
\setcounter{page}{1}

% ----------------------------------------------------------------------------------------------------------
% Einleitung
% ----------------------------------------------------------------------------------------------------------
\section{Einleitung}


\subsection{Motivation}
// wieso weshalb warum wo\\
// Beschreibung abatAG\\
// Enstehung des Projekts\\

\subsection{Problemstellung}
// aktuelle situationsbeschreibung\\
// was soll besser laufen\\

\pagebreak
\subsection{Zielsetzung}
// Das Produkt - Template Programmierung für SAP Frontends mit SAP UI5

\subsection{Struktur}
// der weg über die software ergonomie und ihre wichtigkeit, gezeigt über die Marktanalyse, hin zur praktischen Umsetzung durch Grundlagen und Beschreibung des Lösungsweges\\

\pagebreak

% ----------------------------------------------------------------------------------------------------------
% Kapitel
% ----------------------------------------------------------------------------------------------------------
\section{Software Ergnomie}
// Beleg für die Wichtigkeit von Software Ergonomie\\
// Kurze Übersicht über das Themenfeld Software Ergonomie\\
// Wichtigsten Aspekte nennen und näher erläutern\\

\subsection{Definition}
\subsubsection{Kognitionspsychologie}
// Modellierung und Simulation von menschlichen Denk- und Wahrnehmungsprozessen\\

\subsubsection{Arbeitsphysiologie, Industrieanthropologie}
// Beschäftigung mit grundlegenden menschlichen Fähigkeiten zur Informationsaufnahme und Informationsverarbeitung\\

\subsubsection{Arbeitspsychologie}
// Untersuchung der Wechselbeziehungen zwischen Arbeit, deren Schnittstellen und psychischen Faktoren (unter anderem Arbeitszufriedenheit und -unlust)\\

\subsection{Marktanalyse}
\subsubsection{Nicht-SAP Lösung}
// Oracle oder andere ERP Anbieter\\
// jedoch müssen auch die Anforderungen von Claas abgedeckt sein\\

\subsubsection{SAP Standard Lösung}
// kann der SAP Standard das Abbilden was Claas haben will\\

\subsection{SAP Technologien in Bezug auf Software Ergonomie}
\subsubsection{Business Server Pages}
// \ac{BSP} ist old school Technik\\
// geklaut von \ac{JSP}\\

\subsubsection{Web Dynpro for ABAP}
// Aktuelle Technik\\
// ABAP Code generiert HTML\\
// statischer und dynamischer Teil\\

\subsubsection{SAP Fiori / SAP UI5 / SAP Screen Personas}
// cutting edge\\
// aktuelle SAP UI Strategie\\
// SAP Präsi Chart Fiori/SP renew, etc. pp\\
// SAP Fiori einerseits Name des Themes/Guideline\\
// andererseits Bündel der gängigsten TAs/GPs als fertige\\
// Mobile First/Responsive Design Applikationen\\
// SAP UI5 - SAPs Framework zur Entwicklung von eigenen Applikationen im Fiori Style\\
// Nicht zu tief auf JS, HTML etc eingehen, dass kommt im nächsten Kapitel\\
// SAP SP - Zusätzliche Schicht um Standard Dynpro zu Personalisieren und so\\

\pagebreak

% ----------------------------------------------------------------------------------------------------------
% Kapitel
% ----------------------------------------------------------------------------------------------------------
\section{Praktische Umsetzung}
Lorem ipsum dolor sit amet.

\subsection{Darstellung und Analyse}
// Frontend, Backend, Analyse der wichtigen Arbeitschritte\\
// Anbindung von \ac{SPP} als Datenquelle ansprechen\\

\subsection{Grundlagen}
\subsubsection{HTML und CSS}
// Dokument Grundgerüst, Wichtige Sprachelemente\\
// Allgemeiner Aufbau, Selektoren\\

\subsubsection{JavaScript}
// Grundlagen, Variablen, Operatoren\\
// Kontrollstrukturen, Document Object Model, Ereignisse\\
// jQuery, Selektoren, Ereignisse, DOM-Manipulation, AJAX\\

\subsubsection{SAP UI5 Framework}
\paragraph{Definition}
$\;$ \\
// Aufbauend auf jQuery, AJAX, HTML5/CSS3 \cite{AntoEinf2014}\\

\paragraph{Architektur}
$\;$ \\
// Einführung in SAPUI5 S. 123 \cite{sapui5}\\

\paragraph{OData Protokoll}
$\;$ \\
// Einführung in SAPUI5 S. 168 \\
// SAP Netweaver Gateway OData Services\\

\subsubsection{ABAP}
// Grundlagen, Herkunft/Entstehung\\
// Wichtige Elemente (OpenSQL)\\

\subsection{Lösungsschritte}
\subsubsection{Entwicklungsumgebung}
// Kurze Beschreibung der Entwicklungsumgebung\\
// Sprich Eclipse, SE80, Chrome Dev-tools\\

\subsubsection{UI Design und Prototyping}
// Wireframing als Prototyping\\

\subsubsection{PLATZHALTER}
// HIER KÖNNTE IHRE WERBUNG STEHEN \cite{online}\\

\subsection{Lösung}
\subsubsection{View}
// Auszugsweise Coding bringen um bestimmte Elemente aus der Theorie zu zeigen\\
// Listings lassen sich im Text referenzieren: Listing \ref{lst:app.view.js}\\

\vspace{1em}
\begin{lstlisting}[caption=Root View der Applikation, label=lst:app.view.js]
sap.ui.jsview("abat.Mockup.view.App", {

	getControllerName: function () {
		return "abat.Mockup.view.App";
	},
	
	createContent: function (oController) {
		
		// to avoid scroll bars on desktop the root view must be set to block display
		this.setDisplayBlock(true);
		
		// create app
		this.app = new sap.m.SplitApp();

		// load the master page
		var master = sap.ui.xmlview("Master", "abat.Mockup.view.Master");
		master.getController().nav = this.getController();
		this.app.addPage(master, true);
		
		// load the empty page
		var empty = sap.ui.xmlview("Empty", "abat.Mockup.view.Empty");
		this.app.addPage(empty, false);
		
		// wrap app with shell
		return new sap.m.Shell("Shell", {
			title : "{i18n>ShellTitle}",
			showLogout : false,
			app : this.app
		});
	}
});
\end{lstlisting}

\subsubsection{Model und Controller}
// die Verbindung von beiden Anhand von Coding zeigen\\

\subsubsection{Backend}
// ABAP Stack der den RESTful Service bereitstellt zeigen\\
// Implementation des HTTP Responses\\

\subsubsection{Test}
// Test durch Dummy Daten\\

\pagebreak

% ----------------------------------------------------------------------------------------------------------
% Kapitel
% ----------------------------------------------------------------------------------------------------------
\section{Schluss}
Lorem ipsum dolor sit amet.

\subsection{Zusammenfassung}
// Arbeitsgebiete, Produktions \& Dienstleistungsbereiche\\
// Arbeitsergebnisse\\
// Projektziele, Projektergebnisse, Projekttermine\\
// Mitwirkungszeiträume\\
// Liste aller selbst wahrgenommen Aufgaben und Tätigkeiten\\
// Projektmeilensteine\\
// Ablauforganisation \& Beteiligte\\
// Arbeitsformen, Arbeitsmittel, Arbeitsabläufe\\
// Kommunikations- / Informationsgewohnheiten\\
// Auswertung relevanter Literatur\\
// Themen aus Lehrveranstaltungen\\

\subsection{Bewertung}
// Wesentliche Erkenntnisse und Erfahrungen\\
// Folgerungen und Konsequenzen\\
// Vorschläge für Verbesserung und Veränderung\\
// Auswirkungen auf persönliche Berufs- und Karriereplanung\\
// Bezug zum Studium\\
// hilfreiche Studieninhalte\\
// neu gewonnenes Interesse\\
\pagebreak

%
----------------------------------------------------------------------------------------------------------
% Literatur
% ----------------------------------------------------------------------------------------------------------
\renewcommand\refname{Quellenverzeichnis}
\bibliographystyle{myalpha}
\bibliography{bibo}
\pagebreak

% ----------------------------------------------------------------------------------------------------------
% Anhang
% ----------------------------------------------------------------------------------------------------------
\pagenumbering{Roman}
\setcounter{page}{1}
\lhead{Anhang \thesection}

\begin{appendix}
\section*{Anhang}
\phantomsection
\addcontentsline{toc}{section}{Anhang}
\addtocontents{toc}{\vspace{-0.5em}}

\section{GUI}
Ein toller Anhang.

\subsection*{Screenshot}
\label{app:screenshot}
Unterkategorie, die nicht im Inhaltsverzeichnis auftaucht.

\end{appendix}

\newpage
\thispagestyle{empty}
\begin{center}
	\vspace*{5em}
	\huge\textbf{Erklärung}\\
\end{center}
\vspace{2em}
Hiermit versichere ich, dass ich meine Abschlussarbeit selbständig verfasst und keine anderen als die angegebenen Quellen und Hilfsmittel benutzt habe.

\vspace{4em}
\begin{minipage}{\linewidth}
	\begin{tabular}{p{15em}p{15em}}
		Datum: &  .......................................................\\
		& \centering (Unterschrift)\\
	\end{tabular}
\end{minipage}

\end{document}
