\section*{Zusammenfassung}
Diese Arbeit beschäftigt sich mit der prototypischen Implementierung einer SAP UI5-Applikation im SAP Umfeld und einer zusätzlichen Analyse eines effizienten Einsatzes von UI-Objekten. Zu Beginn werden die Grundlagen näher gebracht, es wird ein Verständnis der verwendeten Techniken aufgebaut. Die Programmiersprachen und Methoden innerhalb des SAP UI5-Frameworks werden erläutert. Dieser Erläuterung folgt die Implementierung des Prototypen. Dabei wird auf die Schlüsselpunkte des Anwendungskonstrukts eingegangen, um die programmatischen Zusammenhänge einer SAP UI5-Anwendung zu zeigen. Die Implementierung endet mit dem Ausliefern der Applikation auf einem SAP Applikations Server, der als Backend Server fungiert. Im Anschluss an die Implementierung wird der Prototyp einer Analyse unterzogen. Diese Analyse soll den Einsatz der UI-Objekte auf Effizienz prüfen. Dabei wird aus den Versuchsergebnissen die Erkenntnis gezogen, dass eine SAP UI5-Anwendung aus Sicht der Performance einer herkömmlichen Desktop Anwendung in nichts nachsteht.

%\begin{verbatim}

%

%\end{verbatim}

\section*{Abstract}
This thesis describes the process of implementation and analysis of a SAP UI5 application. The fundamental technologies, which are used by the sap ui5 framework, are explained in depth at the beginning. The programming languages and methods that are working inside the sap ui5 framework are described. This description follows the implementation of the prototype application. Key aspects of this prototype are shown to clarify connections between different parts of this application. The chapter closes with the deployment of the prototype on a SAP application server. After this the prototype is analysed. The analysis show the efficient use of user interface objects. Therefore the test results verify that a sap ui5 application in a browser environment could easily be compared in aspects of performance with a desktop developed application like standard sap gui.