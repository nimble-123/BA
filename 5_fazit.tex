\section{Fazit}\label{fazit}
Zusammenfassend lässt sich behaupten, dass mit dem SAP UI5 Framework relativ schnell Applikationen entwickelt werden können. Hinsichtlich aktueller Web Standards und Design Richtlinien erfüllen die, mit dem Framework entwickelten, Applikationen sämtliche Kriterien. HTML5, CSS3 und JavaScript werden so eingesetzt, dass optisch ansprechende Anwendungen entstehen, die zudem auch aus Performance Sicht mithalten können im Vergleich zu herkömmlichen Desktop Anwendungen. Diese Applikationen sind als Client Software zu verstehen, im Hintergrund arbeitet als Server weiterhin ein SAP Application Server. Im Rahmen der Arbeit wurde die prototypische Applikation nur mit Testdaten einer Zeitmessung unterzogen. Dabei zeigte sich, dass das UI in unter einer Sekunde für den Anwender fertig gerendert und angezeigt wird. Interessant wäre selbiger Versuch in einem realen produktiven Umfeld. Dort würde sich zeigen ob eine SAP UI5 Applikation tatsächlich in Punkten der Performance mit einer Transaktion im Standard SAP GUI eben würdig ist. Bei der Optik haben SAP UI5 Applikationen deutlich gegenüber dem Standard SAP GUI gewonnen.