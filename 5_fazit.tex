\section{Fazit}\label{fazit}
Zusammenfassend lässt sich sagen, dass mit dem SAP-UI5-Framework schnell Applikationen entwickelt werden können. Hinsichtlich aktueller Webstandards und Designrichtlinien erfüllen die, mit dem Framework entwickelten, Applikationen sämtliche Kriterien. HTML5, CSS3 und JavaScript werden so eingesetzt, dass optisch ansprechende Anwendungen entstehen, die zudem auch aus Sicht der Perfomance im Vergleich zu herkömmlichen Desktop Anwendungen mithalten können. Diese Applikationen sind als Client-Software zu verstehen, im Hintergrund arbeitet als Server weiterhin ein SAP Application Server.\par Im Rahmen der Arbeit wurde die prototypische Applikation nur mit Testdaten einer Zeitmessung unterzogen. Dabei zeigte sich, dass das UI in unter einer Sekunde für den Anwender fertig gerendert und angezeigt wird. Interessant wäre selbiger Versuch in einem realen produktiven Umfeld. Dort würde sich zeigen, ob eine SAP-UI5-Applikation tatsächlich in Punkten der Performance mit einer Transaktion im Standard SAP GUI ebenbürtig ist. Bei der Optik haben SAP-UI5-Applikationen deutlich gegenüber dem Standard SAP GUI gewonnen.\par Aktuellen Presse-Mitteilungen der SAP AG ist auch zu entnehmen, dass SAP-UI5-Applikationen definitiv die Zukunft der SAP ERP Software sein werden. So wurde mit der Präsentation der neuen S/4HANA Suite der Weg bereitet für eine Vielzahl an neuen innovativen Applikationen, die durch ein SAP System als Basis gestützt sind.(vgl. \cite{cbs4hana})